% Options for packages loaded elsewhere
\PassOptionsToPackage{unicode}{hyperref}
\PassOptionsToPackage{hyphens}{url}
%
\documentclass[
]{article}
\usepackage{amsmath,amssymb}
\usepackage{iftex}
\ifPDFTeX
  \usepackage[T1]{fontenc}
  \usepackage[utf8]{inputenc}
  \usepackage{textcomp} % provide euro and other symbols
\else % if luatex or xetex
  \usepackage{unicode-math} % this also loads fontspec
  \defaultfontfeatures{Scale=MatchLowercase}
  \defaultfontfeatures[\rmfamily]{Ligatures=TeX,Scale=1}
\fi
\usepackage{lmodern}
\ifPDFTeX\else
  % xetex/luatex font selection
\fi
% Use upquote if available, for straight quotes in verbatim environments
\IfFileExists{upquote.sty}{\usepackage{upquote}}{}
\IfFileExists{microtype.sty}{% use microtype if available
  \usepackage[]{microtype}
  \UseMicrotypeSet[protrusion]{basicmath} % disable protrusion for tt fonts
}{}
\makeatletter
\@ifundefined{KOMAClassName}{% if non-KOMA class
  \IfFileExists{parskip.sty}{%
    \usepackage{parskip}
  }{% else
    \setlength{\parindent}{0pt}
    \setlength{\parskip}{6pt plus 2pt minus 1pt}}
}{% if KOMA class
  \KOMAoptions{parskip=half}}
\makeatother
\usepackage{xcolor}
\usepackage[margin=1in]{geometry}
\usepackage{color}
\usepackage{fancyvrb}
\newcommand{\VerbBar}{|}
\newcommand{\VERB}{\Verb[commandchars=\\\{\}]}
\DefineVerbatimEnvironment{Highlighting}{Verbatim}{commandchars=\\\{\}}
% Add ',fontsize=\small' for more characters per line
\usepackage{framed}
\definecolor{shadecolor}{RGB}{248,248,248}
\newenvironment{Shaded}{\begin{snugshade}}{\end{snugshade}}
\newcommand{\AlertTok}[1]{\textcolor[rgb]{0.94,0.16,0.16}{#1}}
\newcommand{\AnnotationTok}[1]{\textcolor[rgb]{0.56,0.35,0.01}{\textbf{\textit{#1}}}}
\newcommand{\AttributeTok}[1]{\textcolor[rgb]{0.77,0.63,0.00}{#1}}
\newcommand{\BaseNTok}[1]{\textcolor[rgb]{0.00,0.00,0.81}{#1}}
\newcommand{\BuiltInTok}[1]{#1}
\newcommand{\CharTok}[1]{\textcolor[rgb]{0.31,0.60,0.02}{#1}}
\newcommand{\CommentTok}[1]{\textcolor[rgb]{0.56,0.35,0.01}{\textit{#1}}}
\newcommand{\CommentVarTok}[1]{\textcolor[rgb]{0.56,0.35,0.01}{\textbf{\textit{#1}}}}
\newcommand{\ConstantTok}[1]{\textcolor[rgb]{0.00,0.00,0.00}{#1}}
\newcommand{\ControlFlowTok}[1]{\textcolor[rgb]{0.13,0.29,0.53}{\textbf{#1}}}
\newcommand{\DataTypeTok}[1]{\textcolor[rgb]{0.13,0.29,0.53}{#1}}
\newcommand{\DecValTok}[1]{\textcolor[rgb]{0.00,0.00,0.81}{#1}}
\newcommand{\DocumentationTok}[1]{\textcolor[rgb]{0.56,0.35,0.01}{\textbf{\textit{#1}}}}
\newcommand{\ErrorTok}[1]{\textcolor[rgb]{0.64,0.00,0.00}{\textbf{#1}}}
\newcommand{\ExtensionTok}[1]{#1}
\newcommand{\FloatTok}[1]{\textcolor[rgb]{0.00,0.00,0.81}{#1}}
\newcommand{\FunctionTok}[1]{\textcolor[rgb]{0.00,0.00,0.00}{#1}}
\newcommand{\ImportTok}[1]{#1}
\newcommand{\InformationTok}[1]{\textcolor[rgb]{0.56,0.35,0.01}{\textbf{\textit{#1}}}}
\newcommand{\KeywordTok}[1]{\textcolor[rgb]{0.13,0.29,0.53}{\textbf{#1}}}
\newcommand{\NormalTok}[1]{#1}
\newcommand{\OperatorTok}[1]{\textcolor[rgb]{0.81,0.36,0.00}{\textbf{#1}}}
\newcommand{\OtherTok}[1]{\textcolor[rgb]{0.56,0.35,0.01}{#1}}
\newcommand{\PreprocessorTok}[1]{\textcolor[rgb]{0.56,0.35,0.01}{\textit{#1}}}
\newcommand{\RegionMarkerTok}[1]{#1}
\newcommand{\SpecialCharTok}[1]{\textcolor[rgb]{0.00,0.00,0.00}{#1}}
\newcommand{\SpecialStringTok}[1]{\textcolor[rgb]{0.31,0.60,0.02}{#1}}
\newcommand{\StringTok}[1]{\textcolor[rgb]{0.31,0.60,0.02}{#1}}
\newcommand{\VariableTok}[1]{\textcolor[rgb]{0.00,0.00,0.00}{#1}}
\newcommand{\VerbatimStringTok}[1]{\textcolor[rgb]{0.31,0.60,0.02}{#1}}
\newcommand{\WarningTok}[1]{\textcolor[rgb]{0.56,0.35,0.01}{\textbf{\textit{#1}}}}
\usepackage{graphicx}
\makeatletter
\def\maxwidth{\ifdim\Gin@nat@width>\linewidth\linewidth\else\Gin@nat@width\fi}
\def\maxheight{\ifdim\Gin@nat@height>\textheight\textheight\else\Gin@nat@height\fi}
\makeatother
% Scale images if necessary, so that they will not overflow the page
% margins by default, and it is still possible to overwrite the defaults
% using explicit options in \includegraphics[width, height, ...]{}
\setkeys{Gin}{width=\maxwidth,height=\maxheight,keepaspectratio}
% Set default figure placement to htbp
\makeatletter
\def\fps@figure{htbp}
\makeatother
\setlength{\emergencystretch}{3em} % prevent overfull lines
\providecommand{\tightlist}{%
  \setlength{\itemsep}{0pt}\setlength{\parskip}{0pt}}
\setcounter{secnumdepth}{-\maxdimen} % remove section numbering
\ifLuaTeX
  \usepackage{selnolig}  % disable illegal ligatures
\fi
\IfFileExists{bookmark.sty}{\usepackage{bookmark}}{\usepackage{hyperref}}
\IfFileExists{xurl.sty}{\usepackage{xurl}}{} % add URL line breaks if available
\urlstyle{same}
\hypersetup{
  pdftitle={R\_intro\_Monte\_Carlo},
  hidelinks,
  pdfcreator={LaTeX via pandoc}}

\title{R\_intro\_Monte\_Carlo}
\author{}
\date{\vspace{-2.5em}}

\begin{document}
\maketitle

\hypertarget{play-with-a-sequence}{%
\section{play with a sequence}\label{play-with-a-sequence}}

\begin{Shaded}
\begin{Highlighting}[]
\NormalTok{A\_vector }\OtherTok{=} \FunctionTok{rep}\NormalTok{(}\ConstantTok{NA}\NormalTok{, }\DecValTok{1001}\NormalTok{)}
\NormalTok{A\_vector[}\DecValTok{1}\NormalTok{] }\OtherTok{=} \DecValTok{1}
\NormalTok{A\_vector[}\DecValTok{2}\NormalTok{] }\OtherTok{=} \DecValTok{1}
\ControlFlowTok{for}\NormalTok{(n }\ControlFlowTok{in} \DecValTok{2}\SpecialCharTok{:}\DecValTok{1000}\NormalTok{)\{}
\NormalTok{  A\_vector[n}\SpecialCharTok{+}\DecValTok{1}\NormalTok{] }\OtherTok{=} \DecValTok{2} \SpecialCharTok{*}\NormalTok{ A\_vector[n] }\SpecialCharTok{+}\NormalTok{ A\_vector[n}\DecValTok{{-}1}\NormalTok{]}
\NormalTok{\}}
\NormalTok{A\_vector[}\DecValTok{201}\NormalTok{]}
\end{Highlighting}
\end{Shaded}

\begin{verbatim}
## [1] 1.795176e+76
\end{verbatim}

\begin{Shaded}
\begin{Highlighting}[]
\NormalTok{A\_vector[}\DecValTok{901}\NormalTok{]}
\end{Highlighting}
\end{Shaded}

\begin{verbatim}
## [1] Inf
\end{verbatim}

\hypertarget{how-to-work-this-out-on-log-scale-to-avoid-inf}{%
\section{how to work this out on log scale to avoid
``Inf''}\label{how-to-work-this-out-on-log-scale-to-avoid-inf}}

\begin{Shaded}
\begin{Highlighting}[]
\NormalTok{log\_A\_vector }\OtherTok{=} \FunctionTok{rep}\NormalTok{(}\ConstantTok{NA}\NormalTok{, }\DecValTok{1001}\NormalTok{)}
\NormalTok{log\_A\_vector[}\DecValTok{1}\NormalTok{] }\OtherTok{=} \DecValTok{0}
\NormalTok{log\_A\_vector[}\DecValTok{2}\NormalTok{] }\OtherTok{=} \DecValTok{0}
\ControlFlowTok{for}\NormalTok{(n }\ControlFlowTok{in} \DecValTok{2}\SpecialCharTok{:}\DecValTok{1000}\NormalTok{)\{}
\NormalTok{  temp }\OtherTok{\textless{}{-}} \FunctionTok{max}\NormalTok{(log\_A\_vector[n}\DecValTok{{-}1}\NormalTok{], log\_A\_vector[n])}
\NormalTok{  log\_A\_vector[n}\SpecialCharTok{+}\DecValTok{1}\NormalTok{] }\OtherTok{=}\NormalTok{ temp }\SpecialCharTok{+} \FunctionTok{log}\NormalTok{(}\DecValTok{2} \SpecialCharTok{*} \FunctionTok{exp}\NormalTok{(log\_A\_vector[n] }\SpecialCharTok{{-}}\NormalTok{ temp) }\SpecialCharTok{+} \FunctionTok{exp}\NormalTok{(log\_A\_vector[n}\DecValTok{{-}1}\NormalTok{] }\SpecialCharTok{{-}}\NormalTok{ temp))}
\NormalTok{\}}

\NormalTok{log\_A\_vector[}\DecValTok{201}\NormalTok{]}
\end{Highlighting}
\end{Shaded}

\begin{verbatim}
## [1] 175.5816
\end{verbatim}

\begin{Shaded}
\begin{Highlighting}[]
\NormalTok{log\_A\_vector[}\DecValTok{901}\NormalTok{]}
\end{Highlighting}
\end{Shaded}

\begin{verbatim}
## [1] 792.5431
\end{verbatim}

\hypertarget{vector-and-matrix-operations}{%
\section{Vector and Matrix
operations}\label{vector-and-matrix-operations}}

\begin{Shaded}
\begin{Highlighting}[]
\NormalTok{x }\OtherTok{\textless{}{-}} \FunctionTok{c}\NormalTok{(}\DecValTok{1}\NormalTok{,}\DecValTok{2}\NormalTok{,}\DecValTok{3}\NormalTok{,}\DecValTok{5}\NormalTok{)}
\NormalTok{y }\OtherTok{\textless{}{-}} \FunctionTok{c}\NormalTok{(}\DecValTok{3}\NormalTok{,}\DecValTok{5}\NormalTok{,}\DecValTok{7}\NormalTok{,}\DecValTok{6}\NormalTok{)}
\NormalTok{x}\SpecialCharTok{+}\NormalTok{y}
\end{Highlighting}
\end{Shaded}

\begin{verbatim}
## [1]  4  7 10 11
\end{verbatim}

\begin{Shaded}
\begin{Highlighting}[]
\NormalTok{x}\SpecialCharTok{*}\NormalTok{y}
\end{Highlighting}
\end{Shaded}

\begin{verbatim}
## [1]  3 10 21 30
\end{verbatim}

\begin{Shaded}
\begin{Highlighting}[]
\NormalTok{x}\SpecialCharTok{\%*\%}\FunctionTok{t}\NormalTok{(y)}
\end{Highlighting}
\end{Shaded}

\begin{verbatim}
##      [,1] [,2] [,3] [,4]
## [1,]    3    5    7    6
## [2,]    6   10   14   12
## [3,]    9   15   21   18
## [4,]   15   25   35   30
\end{verbatim}

\begin{Shaded}
\begin{Highlighting}[]
\NormalTok{x }\SpecialCharTok{*} \DecValTok{4}
\end{Highlighting}
\end{Shaded}

\begin{verbatim}
## [1]  4  8 12 20
\end{verbatim}

\begin{Shaded}
\begin{Highlighting}[]
\NormalTok{x}\SpecialCharTok{\^{}}\DecValTok{2}
\end{Highlighting}
\end{Shaded}

\begin{verbatim}
## [1]  1  4  9 25
\end{verbatim}

\hypertarget{r-list-data-frame}{%
\section{R list (data frame)}\label{r-list-data-frame}}

\begin{Shaded}
\begin{Highlighting}[]
\NormalTok{my\_list }\OtherTok{\textless{}{-}} \FunctionTok{list}\NormalTok{(}\AttributeTok{First\_class =} \FunctionTok{c}\NormalTok{(}\DecValTok{1}\NormalTok{, }\DecValTok{2}\NormalTok{, }\DecValTok{3}\NormalTok{, }\DecValTok{4}\NormalTok{, }\DecValTok{4}\NormalTok{, }\DecValTok{4}\NormalTok{), }
                \AttributeTok{Second\_class =} \FunctionTok{matrix}\NormalTok{(}\DecValTok{1}\SpecialCharTok{:}\DecValTok{4}\NormalTok{, }\DecValTok{2}\NormalTok{, }\DecValTok{2}\NormalTok{),}
                \AttributeTok{last\_element =} \FloatTok{2.4}\NormalTok{,}
                \AttributeTok{words =} \FunctionTok{c}\NormalTok{(}\StringTok{"i"}\NormalTok{, }\StringTok{"love"}\NormalTok{, }\StringTok{"statistics"}\NormalTok{))}
\NormalTok{my\_list}\SpecialCharTok{$}\NormalTok{First\_class}
\end{Highlighting}
\end{Shaded}

\begin{verbatim}
## [1] 1 2 3 4 4 4
\end{verbatim}

\begin{Shaded}
\begin{Highlighting}[]
\FunctionTok{names}\NormalTok{(my\_list)}
\end{Highlighting}
\end{Shaded}

\begin{verbatim}
## [1] "First_class"  "Second_class" "last_element" "words"
\end{verbatim}

\hypertarget{sample-random-variables}{%
\section{sample random variables}\label{sample-random-variables}}

\begin{Shaded}
\begin{Highlighting}[]
\CommentTok{\# Coin flip}

\NormalTok{p }\OtherTok{\textless{}{-}} \FloatTok{0.7}
\NormalTok{my\_coin\_flips }\OtherTok{\textless{}{-}} \FunctionTok{rbinom}\NormalTok{(}\DecValTok{100}\NormalTok{, }\DecValTok{1}\NormalTok{, p)}

\CommentTok{\# simulate a Bernoulli random variable }

\NormalTok{p }\OtherTok{\textless{}{-}} \FloatTok{0.7}
\NormalTok{y }\OtherTok{\textless{}{-}} \FunctionTok{rbinom}\NormalTok{(}\DecValTok{100000}\NormalTok{, }\DecValTok{1}\NormalTok{, p)}
\FunctionTok{table}\NormalTok{(y)}
\end{Highlighting}
\end{Shaded}

\begin{verbatim}
## y
##     0     1 
## 29981 70019
\end{verbatim}

\begin{Shaded}
\begin{Highlighting}[]
\CommentTok{\# estimate moment}
\NormalTok{phat }\OtherTok{\textless{}{-}} \FunctionTok{sum}\NormalTok{(y)}\SpecialCharTok{/}\FunctionTok{length}\NormalTok{(y)}
\FunctionTok{print}\NormalTok{(}\FunctionTok{c}\NormalTok{(phat, }\FunctionTok{mean}\NormalTok{(y)))}
\end{Highlighting}
\end{Shaded}

\begin{verbatim}
## [1] 0.70019 0.70019
\end{verbatim}

\begin{Shaded}
\begin{Highlighting}[]
\FunctionTok{print}\NormalTok{(}\FunctionTok{c}\NormalTok{(phat}\SpecialCharTok{*}\NormalTok{(}\DecValTok{1}\SpecialCharTok{{-}}\NormalTok{phat), }\FunctionTok{var}\NormalTok{(y)))}
\end{Highlighting}
\end{Shaded}

\begin{verbatim}
## [1] 0.2099240 0.2099261
\end{verbatim}

\hypertarget{uniform-distribution}{%
\section{uniform distribution}\label{uniform-distribution}}

\begin{Shaded}
\begin{Highlighting}[]
\NormalTok{x }\OtherTok{\textless{}{-}} \FunctionTok{runif}\NormalTok{(}\DecValTok{1000}\NormalTok{, pi, }\DecValTok{2}\SpecialCharTok{*}\NormalTok{pi)}
\FunctionTok{mean}\NormalTok{(x)}
\end{Highlighting}
\end{Shaded}

\begin{verbatim}
## [1] 4.715832
\end{verbatim}

\begin{Shaded}
\begin{Highlighting}[]
\CommentTok{\# true value of the mean}
\NormalTok{(}\DecValTok{3}\SpecialCharTok{*}\NormalTok{pi)}\SpecialCharTok{/}\DecValTok{2}
\end{Highlighting}
\end{Shaded}

\begin{verbatim}
## [1] 4.712389
\end{verbatim}

\begin{Shaded}
\begin{Highlighting}[]
\CommentTok{\# do it for multiple sample sizes}

\ControlFlowTok{for}\NormalTok{(size }\ControlFlowTok{in} \FunctionTok{c}\NormalTok{(}\DecValTok{10}\NormalTok{, }\DecValTok{100}\NormalTok{, }\DecValTok{1000}\NormalTok{, }\DecValTok{1000000}\NormalTok{)) \{}
\NormalTok{  my\_vals }\OtherTok{=} \FunctionTok{runif}\NormalTok{(size, }\AttributeTok{min =}\NormalTok{ pi, }\AttributeTok{max =} \DecValTok{2}\SpecialCharTok{*}\NormalTok{pi)}
  \FunctionTok{print}\NormalTok{(size)}
  \FunctionTok{print}\NormalTok{(}\FunctionTok{mean}\NormalTok{(my\_vals))}
  \FunctionTok{print}\NormalTok{(}\FunctionTok{var}\NormalTok{(my\_vals))}
\NormalTok{\}}
\end{Highlighting}
\end{Shaded}

\begin{verbatim}
## [1] 10
## [1] 4.647034
## [1] 0.6410928
## [1] 100
## [1] 4.732199
## [1] 0.8334222
## [1] 1000
## [1] 4.685734
## [1] 0.8232572
## [1] 1e+06
## [1] 4.7128
## [1] 0.8222763
\end{verbatim}

\begin{Shaded}
\begin{Highlighting}[]
\CommentTok{\# equivalent way of simulating bernoulli experiments}
\NormalTok{x }\OtherTok{\textless{}{-}} \FunctionTok{runif}\NormalTok{(}\DecValTok{10000}\NormalTok{, }\DecValTok{0}\NormalTok{, }\DecValTok{1}\NormalTok{)}
\NormalTok{y }\OtherTok{\textless{}{-}}\NormalTok{ (x }\SpecialCharTok{\textless{}}\NormalTok{ p)}
\FunctionTok{table}\NormalTok{(}\FunctionTok{as.integer}\NormalTok{(x }\SpecialCharTok{\textless{}}\NormalTok{ p))}
\end{Highlighting}
\end{Shaded}

\begin{verbatim}
## 
##    0    1 
## 2906 7094
\end{verbatim}

\hypertarget{what-if-the-coin-is-biased-with-head-prob.-0.4}{%
\section{what if the coin is biased with head prob. =
0.4?}\label{what-if-the-coin-is-biased-with-head-prob.-0.4}}

\hypertarget{gaussian-r.v.s-n0-1}{%
\section{Gaussian r.v.s N(0, 1)}\label{gaussian-r.v.s-n0-1}}

\begin{Shaded}
\begin{Highlighting}[]
\NormalTok{my\_simulation }\OtherTok{\textless{}{-}} \FunctionTok{list}\NormalTok{(}\AttributeTok{ten\_samples =} \FunctionTok{rnorm}\NormalTok{(}\DecValTok{10}\NormalTok{),}
                      \AttributeTok{twen\_samp =} \FunctionTok{rnorm}\NormalTok{(}\DecValTok{20}\NormalTok{), }
                      \AttributeTok{fiftysample =} \FunctionTok{rnorm}\NormalTok{(}\DecValTok{50}\NormalTok{),}
                      \AttributeTok{hundredsamp =} \FunctionTok{rnorm}\NormalTok{(}\DecValTok{100}\NormalTok{), }
                      \AttributeTok{thousampl =} \FunctionTok{rnorm}\NormalTok{(}\DecValTok{1000}\NormalTok{),}
                      \AttributeTok{millionsampl =} \FunctionTok{rnorm}\NormalTok{(}\DecValTok{1000000}\NormalTok{))}

\FunctionTok{sapply}\NormalTok{(my\_simulation, mean)}
\end{Highlighting}
\end{Shaded}

\begin{verbatim}
##   ten_samples     twen_samp   fiftysample   hundredsamp     thousampl 
##  0.2285939007  0.2767953055  0.2433115533 -0.0890364821  0.0153783735 
##  millionsampl 
##  0.0008750301
\end{verbatim}

\begin{Shaded}
\begin{Highlighting}[]
\FunctionTok{sapply}\NormalTok{(my\_simulation, var)}
\end{Highlighting}
\end{Shaded}

\begin{verbatim}
##  ten_samples    twen_samp  fiftysample  hundredsamp    thousampl millionsampl 
##    0.4235398    0.4286027    0.5873172    0.9556188    0.9628816    1.0005573
\end{verbatim}

\begin{Shaded}
\begin{Highlighting}[]
\CommentTok{\# recover density using Monte Carlo Samples}

\FunctionTok{par}\NormalTok{(}\AttributeTok{mfrow =} \FunctionTok{c}\NormalTok{(}\DecValTok{2}\NormalTok{, }\DecValTok{3}\NormalTok{))}
\FunctionTok{hist}\NormalTok{(my\_simulation}\SpecialCharTok{$}\NormalTok{ten\_samples)}
\FunctionTok{hist}\NormalTok{(my\_simulation}\SpecialCharTok{$}\NormalTok{twen\_samp)}
\FunctionTok{hist}\NormalTok{(my\_simulation}\SpecialCharTok{$}\NormalTok{fiftysample)}
\FunctionTok{hist}\NormalTok{(my\_simulation}\SpecialCharTok{$}\NormalTok{hundredsamp)}
\FunctionTok{hist}\NormalTok{(my\_simulation}\SpecialCharTok{$}\NormalTok{thousampl)}

\CommentTok{\# Compute Expectation and Variance of r.v.s using Monte Carlo}

\FunctionTok{print}\NormalTok{(}\FunctionTok{mean}\NormalTok{(my\_simulation}\SpecialCharTok{$}\NormalTok{ten\_samples))}
\end{Highlighting}
\end{Shaded}

\begin{verbatim}
## [1] 0.2285939
\end{verbatim}

\begin{Shaded}
\begin{Highlighting}[]
\FunctionTok{par}\NormalTok{(}\AttributeTok{mfrow =} \FunctionTok{c}\NormalTok{(}\DecValTok{2}\NormalTok{, }\DecValTok{3}\NormalTok{))}
\end{Highlighting}
\end{Shaded}

\includegraphics{R_intro_Monte_Carlo_files/figure-latex/unnamed-chunk-8-1.pdf}

\begin{Shaded}
\begin{Highlighting}[]
\NormalTok{Nsamples\_vec }\OtherTok{\textless{}{-}} \FunctionTok{c}\NormalTok{(}\DecValTok{10}\NormalTok{, }\DecValTok{20}\NormalTok{, }\DecValTok{50}\NormalTok{, }\DecValTok{100}\NormalTok{, }\DecValTok{1000}\NormalTok{, }\DecValTok{2000}\NormalTok{, }\DecValTok{5000}\NormalTok{)}
\ControlFlowTok{for}\NormalTok{(k }\ControlFlowTok{in}\NormalTok{ Nsamples\_vec)\{}
\NormalTok{  mysamples }\OtherTok{\textless{}{-}} \FunctionTok{rnorm}\NormalTok{(k)}
  \FunctionTok{hist}\NormalTok{(mysamples)}
\NormalTok{\}}
\end{Highlighting}
\end{Shaded}

\includegraphics{R_intro_Monte_Carlo_files/figure-latex/unnamed-chunk-8-2.pdf}
\includegraphics{R_intro_Monte_Carlo_files/figure-latex/unnamed-chunk-8-3.pdf}

\begin{Shaded}
\begin{Highlighting}[]
\NormalTok{Nsamples }\OtherTok{\textless{}{-}} \DecValTok{50}
\NormalTok{x }\OtherTok{\textless{}{-}} \FunctionTok{rnorm}\NormalTok{(Nsamples)}
\FunctionTok{hist}\NormalTok{(x)}
\end{Highlighting}
\end{Shaded}

\includegraphics{R_intro_Monte_Carlo_files/figure-latex/unnamed-chunk-9-1.pdf}

\begin{Shaded}
\begin{Highlighting}[]
\FunctionTok{boxplot}\NormalTok{(x)}
\end{Highlighting}
\end{Shaded}

\includegraphics{R_intro_Monte_Carlo_files/figure-latex/unnamed-chunk-9-2.pdf}

\begin{Shaded}
\begin{Highlighting}[]
\NormalTok{xrange }\OtherTok{\textless{}{-}} \FunctionTok{range}\NormalTok{(x)}
\NormalTok{xsteps }\OtherTok{\textless{}{-}} \FunctionTok{seq}\NormalTok{(xrange[}\DecValTok{1}\NormalTok{], xrange[}\DecValTok{2}\NormalTok{], }\AttributeTok{length.out =} \DecValTok{100}\NormalTok{)}
\NormalTok{density\_x }\OtherTok{\textless{}{-}} \FunctionTok{dnorm}\NormalTok{(xsteps)}
\FunctionTok{plot}\NormalTok{(xsteps, density\_x)}
\end{Highlighting}
\end{Shaded}

\includegraphics{R_intro_Monte_Carlo_files/figure-latex/unnamed-chunk-9-3.pdf}

\begin{Shaded}
\begin{Highlighting}[]
\FunctionTok{hist}\NormalTok{(x, }\AttributeTok{freq =} \ConstantTok{FALSE}\NormalTok{)}
\FunctionTok{lines}\NormalTok{(xsteps, density\_x, }\AttributeTok{col =} \StringTok{"red"}\NormalTok{)}
\end{Highlighting}
\end{Shaded}

\includegraphics{R_intro_Monte_Carlo_files/figure-latex/unnamed-chunk-9-4.pdf}

\hypertarget{test-out-the-same-things-with-chi-squared-or-gamma-distribution}{%
\section{test out the same things with chi-squared or gamma
distribution}\label{test-out-the-same-things-with-chi-squared-or-gamma-distribution}}

\hypertarget{law-of-large-numbers}{%
\section{law of large numbers}\label{law-of-large-numbers}}

\begin{Shaded}
\begin{Highlighting}[]
\NormalTok{N }\OtherTok{\textless{}{-}} \DecValTok{1000}
\NormalTok{x }\OtherTok{\textless{}{-}} \FunctionTok{rnorm}\NormalTok{(N, }\DecValTok{4}\NormalTok{, }\DecValTok{2}\NormalTok{)}
\NormalTok{means\_x }\OtherTok{\textless{}{-}} \FunctionTok{cumsum}\NormalTok{(x) }\SpecialCharTok{/}\NormalTok{ (}\DecValTok{1}\SpecialCharTok{:}\NormalTok{N)}
\FunctionTok{plot}\NormalTok{(}\DecValTok{1}\SpecialCharTok{:}\NormalTok{N, means\_x)}
\end{Highlighting}
\end{Shaded}

\includegraphics{R_intro_Monte_Carlo_files/figure-latex/unnamed-chunk-10-1.pdf}

\begin{Shaded}
\begin{Highlighting}[]
\FunctionTok{plot}\NormalTok{(}\DecValTok{1}\SpecialCharTok{:}\DecValTok{40}\NormalTok{, means\_x[}\DecValTok{1}\SpecialCharTok{:}\DecValTok{40}\NormalTok{])}
\end{Highlighting}
\end{Shaded}

\includegraphics{R_intro_Monte_Carlo_files/figure-latex/unnamed-chunk-10-2.pdf}

\hypertarget{central-limit-theorem}{%
\section{central limit theorem}\label{central-limit-theorem}}

\begin{Shaded}
\begin{Highlighting}[]
\NormalTok{y }\OtherTok{\textless{}{-}} \FunctionTok{array}\NormalTok{(}\FunctionTok{rgamma}\NormalTok{(}\DecValTok{1000} \SpecialCharTok{*} \DecValTok{10000}\NormalTok{, }\DecValTok{1}\NormalTok{), }\FunctionTok{c}\NormalTok{(}\DecValTok{1000}\NormalTok{, }\DecValTok{10000}\NormalTok{))}
\FunctionTok{hist}\NormalTok{(y)}
\end{Highlighting}
\end{Shaded}

\includegraphics{R_intro_Monte_Carlo_files/figure-latex/unnamed-chunk-11-1.pdf}

\begin{Shaded}
\begin{Highlighting}[]
\FunctionTok{hist}\NormalTok{(}\FunctionTok{apply}\NormalTok{(y, }\DecValTok{2}\NormalTok{, mean))}
\end{Highlighting}
\end{Shaded}

\includegraphics{R_intro_Monte_Carlo_files/figure-latex/unnamed-chunk-11-2.pdf}

\hypertarget{monte-carlo-integration}{%
\section{Monte Carlo Integration}\label{monte-carlo-integration}}

\begin{Shaded}
\begin{Highlighting}[]
\NormalTok{x }\OtherTok{\textless{}{-}} \FunctionTok{rgamma}\NormalTok{(}\DecValTok{100}\NormalTok{, }\DecValTok{1}\NormalTok{, }\DecValTok{1}\NormalTok{)}

\CommentTok{\# Computing expectation}

\FunctionTok{mean}\NormalTok{(x)}
\end{Highlighting}
\end{Shaded}

\begin{verbatim}
## [1] 0.869031
\end{verbatim}

\begin{Shaded}
\begin{Highlighting}[]
\CommentTok{\# Computing variance}

\FunctionTok{var}\NormalTok{(x)}
\end{Highlighting}
\end{Shaded}

\begin{verbatim}
## [1] 0.924638
\end{verbatim}

\begin{Shaded}
\begin{Highlighting}[]
\CommentTok{\# Computing tail probability}

\FunctionTok{sum}\NormalTok{(x }\SpecialCharTok{\textgreater{}} \DecValTok{3}\NormalTok{) }\SpecialCharTok{/} \FunctionTok{length}\NormalTok{(x)}
\end{Highlighting}
\end{Shaded}

\begin{verbatim}
## [1] 0.05
\end{verbatim}

\hypertarget{regression-models}{%
\subsection{Regression Models}\label{regression-models}}

\begin{Shaded}
\begin{Highlighting}[]
\NormalTok{N }\OtherTok{\textless{}{-}} \DecValTok{100}
\NormalTok{beta }\OtherTok{\textless{}{-}} \DecValTok{1}
\NormalTok{sigma }\OtherTok{=} \FloatTok{0.8}

\NormalTok{x }\OtherTok{\textless{}{-}} \FunctionTok{rnorm}\NormalTok{(N, }\DecValTok{4}\NormalTok{, }\DecValTok{2}\NormalTok{)}
\NormalTok{y }\OtherTok{\textless{}{-}}\NormalTok{ beta }\SpecialCharTok{*}\NormalTok{ x }\SpecialCharTok{+} \FunctionTok{rnorm}\NormalTok{(N) }\SpecialCharTok{*}\NormalTok{ sigma}

\NormalTok{lm\_result }\OtherTok{\textless{}{-}} \FunctionTok{lm}\NormalTok{(y}\SpecialCharTok{\textasciitilde{}}\NormalTok{x)}
\FunctionTok{plot}\NormalTok{(x, y)}
\FunctionTok{lines}\NormalTok{(x, lm\_result}\SpecialCharTok{$}\NormalTok{fitted.values)}
\end{Highlighting}
\end{Shaded}

\includegraphics{R_intro_Monte_Carlo_files/figure-latex/unnamed-chunk-13-1.pdf}

\hypertarget{clustering-models}{%
\subsection{Clustering Models}\label{clustering-models}}

\begin{Shaded}
\begin{Highlighting}[]
\NormalTok{N }\OtherTok{\textless{}{-}} \DecValTok{1000}
\NormalTok{cluster }\OtherTok{\textless{}{-}} \FunctionTok{sample}\NormalTok{(}\FunctionTok{c}\NormalTok{(}\DecValTok{1}\NormalTok{, }\DecValTok{2}\NormalTok{, }\DecValTok{3}\NormalTok{, }\DecValTok{4}\NormalTok{), N, }\AttributeTok{replace =} \ConstantTok{TRUE}\NormalTok{, }\AttributeTok{prob =} \FunctionTok{c}\NormalTok{(}\FloatTok{0.1}\NormalTok{, }\FloatTok{0.2}\NormalTok{, }\FloatTok{0.1}\NormalTok{, }\FloatTok{0.6}\NormalTok{))}
\NormalTok{mu }\OtherTok{\textless{}{-}} \FunctionTok{c}\NormalTok{(}\DecValTok{1}\NormalTok{, }\DecValTok{2}\NormalTok{, }\DecValTok{3}\NormalTok{, }\DecValTok{4}\NormalTok{)}
\NormalTok{sigma }\OtherTok{\textless{}{-}} \FunctionTok{rep}\NormalTok{(}\FloatTok{0.1}\NormalTok{, }\DecValTok{4}\NormalTok{)}
\NormalTok{y }\OtherTok{\textless{}{-}} \FunctionTok{rnorm}\NormalTok{(N) }\SpecialCharTok{*}\NormalTok{ sigma[cluster] }\SpecialCharTok{+}\NormalTok{ mu[cluster]}
\FunctionTok{hist}\NormalTok{(y, }\DecValTok{50}\NormalTok{)}
\end{Highlighting}
\end{Shaded}

\includegraphics{R_intro_Monte_Carlo_files/figure-latex/unnamed-chunk-14-1.pdf}

\hypertarget{quiz-game}{%
\subsection{QUIZ game}\label{quiz-game}}

\begin{Shaded}
\begin{Highlighting}[]
\NormalTok{probquiz }\OtherTok{\textless{}{-}} \FunctionTok{runif}\NormalTok{(}\DecValTok{1}\NormalTok{)}
\CommentTok{\#probquiz \textless{}{-} sample(0.1*(1:6), 1, replace = TRUE, prob = rep(1, 6))}
\NormalTok{B }\OtherTok{\textless{}{-}} \FunctionTok{rbinom}\NormalTok{(}\AttributeTok{n =} \DecValTok{1}\NormalTok{, }\AttributeTok{size =} \DecValTok{1}\NormalTok{, }\AttributeTok{prob =}\NormalTok{ probquiz)}
\FunctionTok{print}\NormalTok{(}\FunctionTok{list}\NormalTok{(}\FunctionTok{as.integer}\NormalTok{(B), probquiz))}
\end{Highlighting}
\end{Shaded}

\begin{verbatim}
## [[1]]
## [1] 0
## 
## [[2]]
## [1] 0.7469673
\end{verbatim}

\hypertarget{exercise-if-we-have-26-lectures-during-the-semester-what-is-the-expected-number-of-quizzes-what-is-the-corresponding-standard-deviation}{%
\section{Exercise: if we have 26 lectures during the semester, what is
the expected number of quizzes? What is the corresponding standard
deviation?}\label{exercise-if-we-have-26-lectures-during-the-semester-what-is-the-expected-number-of-quizzes-what-is-the-corresponding-standard-deviation}}

\begin{Shaded}
\begin{Highlighting}[]
\NormalTok{quizzz }\OtherTok{\textless{}{-}} \FunctionTok{rbinom}\NormalTok{(}\DecValTok{1000}\NormalTok{, }\DecValTok{26}\NormalTok{, }\FloatTok{0.5}\NormalTok{)}
\NormalTok{quizzz}
\end{Highlighting}
\end{Shaded}

\begin{verbatim}
##    [1] 14 17 11 10 13 14  9 10 16 15 10 11 16 15 13 12 13 14 14 15 12  9 14 15
##   [25] 15 12 16 10 14 11 15 14 17 10 16 12 17 15 17 17 11 14 14 15 16 12 15 13
##   [49] 13 12 13  9 12 12 12  9 10 11 11 18  9 12 17 12 17 11 10 18 11 11 11 13
##   [73] 14  9 12 17 13  9 16 15 14 11 16 13 16 17 13 15 13 13 10 11 12 15 12 12
##   [97]  9  9 11 15 12 11 14 13 15 12 13 18 17 15 19 10 12 14 11 10 19 13 10 12
##  [121] 13 11  7 17 17 11 16 10 11 12 15 16 15 14 15 15 17 11 16 14 17 10 14 13
##  [145] 14 13 15 15 14 14 10 11 14 11  7 13 10 14 12 11 12 14 12 16 14 16  8 16
##  [169] 11 12 18 11 12 14 17 13 15 15 14 12 16 12 13  8 14 13 14 10 13 12 16 15
##  [193] 10 15 14 18 15 15 13 15 11 14  9 17 14 16 13 17 15 12 13 15 11 13 14  5
##  [217] 11 10 14  9 10 13 10 17 10  9 12 15 14 14 14 13 13 10 13 13 17 16 14 14
##  [241]  8  9 20 14 11 13 16 13 11 15 17 19 12 13 17 14 14 13 15 10 11 15 13 13
##  [265] 16  9 11 12 10 11 16 13 10 13  9 14 13 10 10 12 11 16 13 13 14  9 13 16
##  [289] 12 12 13 10 10 10 12 14 15 14 15 15 14 18 15 14 10 11 13 11 11 12 13 13
##  [313] 13 10 10 14 13 13 15 14 11 14 12 14 10 13 12 10 11 15 11 10 16 12 12 15
##  [337] 16 14 16 10 12  9 13 10 19 12 12 14  9 10  9 14 12 13 13 11 17 13 13 12
##  [361] 11 12  9 15 11 10 10 15 13 13 13 16 15 14  8 16 12 17 14  9 13 12 13 17
##  [385] 15  9  9  7 16 16 12 11 11 13 12 13 13 14 11 11 11  8 10 15 10 13 18 16
##  [409] 11 14 14 14 13 12 12 10 14 17 10 13 13 11 10  6 11 13 15 10 13 15 14 10
##  [433] 10 13 13 14  8 13 13 12 10 13 13 15 13  9 11 11 17 14 17 13 14 10 16 12
##  [457] 17 15 17 14 17 13 12 10 16 13 14 12 13  9 19 17 14  9 15 15 15 12 11 12
##  [481] 13 14 16 12  9 13 14 15 17 18 13 12 14 11 13 12 12 14 16 17 14 16 13 10
##  [505] 12 18 19 11 11 11 11 16 16 14 13 14 14 12 14 14 14 13 10 14 16 13 15 10
##  [529] 13 15 15 14 13  8 17 13  9 14 10 12 12 11 16  8 16 14 12 17 15 13 13  9
##  [553] 19 16 16 12 14 19 16 12 14 12 13  9 15 12 14 13 12 20 16 14 16 16 14 13
##  [577] 13 13 15 16 14 15 16 10 13 10 11 10 11 10 16 15  8 17 10 15 11 12 15 10
##  [601] 14  5 13 10 11 11 10 12 10 13 14 14 12 12  8 12 13 12 18 12 10 10 12 16
##  [625] 14 17 15 15 13 13 10 10 11 10 12  9 14  8  8 11 17 14 10 15 13 14 13 14
##  [649] 12 17 13 13 12  9 10 12 14 14 15 12 11 12  8 14 19  8 14 12 17 17 10 18
##  [673] 13 10 12 15 10 11 11 12 13 13  9 15 12 12 12 10 16 16 15  5 10 17 10 12
##  [697] 13 14 14 11 13 14 12 16 11 13 17 13 14 16 13 20  9 14 15 14 16 14 13  3
##  [721] 14 10 11 12 11 16 14  8 12 11 11 11 18 14 15 14  9 15 11 11 11 11 12 12
##  [745] 11 16 13  8 16 18 16 13 12 12 18 11 11 10 16 13 12 14 11 15  7 14 19 14
##  [769] 18  8  9 15 13 18 12 14 13 13 17  7 13  7 13 15 16 13 16  9 17 12 18 13
##  [793] 12 13 15 14 10 12 12 11 16 15 13 15 17 13 11 10 14 11 13 15 12 11 13 13
##  [817] 13 10 16 13  9 12 17  9 17 16 17 12 13 10 11 17 10 12 14 15 13 13 11 18
##  [841] 15 12 13 13 14 12 13 12 11 11 11 13 11 16 12 10  9 15 14 13 16  9 10 11
##  [865] 13 13  8 14 14 18 17 18 15 10 17 15 18 14 12 13 12  8 11 15 12 13 14 10
##  [889] 16 14 13 14  7 11 13 14 12 16 13 15 16 17 14  9 14 14 15 15 12 10  9 12
##  [913] 20 13 17 14 12 17 13 12 15 15 11 17 10 17 15 11 13 11 12 16 10  9 12 17
##  [937] 12 16 16 14 15 13 17  9 14 12 15 13  9 12 15 13 13 12 11 13 11 10 14 13
##  [961] 17  8 12 10  9  7 17 14 12 15  7 14 13 15 18  7 10 12 10  7 12 18 15 16
##  [985] 16 12 18 13 12 14 10 14 12 14 18 11 12 17 10 15
\end{verbatim}

\begin{Shaded}
\begin{Highlighting}[]
\FunctionTok{mean}\NormalTok{(quizzz)}
\end{Highlighting}
\end{Shaded}

\begin{verbatim}
## [1] 12.996
\end{verbatim}

\begin{Shaded}
\begin{Highlighting}[]
\FunctionTok{var}\NormalTok{(quizzz)}
\end{Highlighting}
\end{Shaded}

\begin{verbatim}
## [1] 6.822807
\end{verbatim}

\hypertarget{exercise-can-you-verify-the-results-above-theoretically}{%
\section{Exercise: can you verify the results above
theoretically?}\label{exercise-can-you-verify-the-results-above-theoretically}}

\end{document}
